\usepackage[left=1.00in,top=1.00in,right=1.00in,bottom=1.00in]{geometry}
\usepackage{calc}
\usepackage{fancyhdr}
\usepackage{amsmath}
\usepackage{amssymb}
\usepackage{amsfonts}
\usepackage{graphicx}
\usepackage[linesnumbered,boxed]{algorithm2e}
\usepackage{amsthm}
\usepackage{amscd}
\usepackage{eufrak}
\usepackage{amsmath}
\usepackage{hyperref} 
\usepackage{lscape}
\usepackage{morefloats}
\usepackage{stmaryrd}


%\usepackage{amssymb}
%\usepackage{amsmath}
\usepackage{diagrams}
\usepackage{color}
\usepackage{subfigure}
%\diagramstyle[labelstyle=\scriptstyle]
\newarrow{Dto}{<}{-}{-}{-}{>}
\usepackage[all]{xy}
\diagramstyle[labelstyle=\scriptstyle]
%\usepackage{a4wide} 														  	% Iets meer tekst op een bladzijde
%% The amsthm package provides extended theorem environments
%% \usepackage{amsthm}

%% The numcompress package shorten the last page in references.
%% `nodots' option removes dots from firstnames in references.
\usepackage[nodots]{numcompress}

\newtheorem{theorem}{Theorem}[section]
\newtheorem{proposition}{Proposition}[section]
\newtheorem{lemma}{Lemma}[section]
\newtheorem{corollary}{Corollary}[section]
\newtheorem{remark}{Remark}[section]
\newtheorem{example}{Example}[section]
\newtheorem{definition}{Definition}[section]
\newtheorem{conjecture}{Conjecture}[section]
\newtheorem{balance}{Balance Law}[section]

\DeclareMathAlphabet{\mathpzc}{OT1}{pzc}{m}{it}


\newcommand{\basisvector}[2]{\ensuremath{\text{\mbox{\small $\frac{\boldsymbol{\partial}}{\boldsymbol{\partial} #1^{#2}}$}}}}

% vector and form definition
\newcommand{\param}{\ensuremath{\xi}}
\newcommand{\vect}[2]{\ensuremath{\boldsymbol{\mathrm{#1}}_{#2}}}
\newcommand{\Tvect}[2]{\ensuremath{#1_{#2}}}
\newcommand{\Dform}[2]{\ensuremath{#1^{#2}}}

% cochain and chain definition
\newcommand{\chain}[2]{\ensuremath{\vect{#1}{#2}}}
\newcommand{\cochain}[2]{\ensuremath{#1^{#2}}}
\newcommand{\supp}[1]{\ensuremath{\text{supp}\left\{#1\right\}}}

\newcommand{\boundary}{\ensuremath{\partial}}
\newcommand{\coboundary}{\ensuremath{\delta}}

% coordinates
\newcommand{\coord}[2]{\ensuremath{#1_{#2}}}
\newcommand{\x}[1]{\ensuremath{x^{#1}}}
\newcommand{\y}[1]{\ensuremath{y^{#1}}}
\newcommand{\dx}[1]{\ensuremath{\mathrm{d}x^{#1}}}
\newcommand{\dy}[1]{\ensuremath{\mathrm{d}y^{#1}}}	
\newcommand{\dA}[2]{\ensuremath{\mathrm{\dx{#1} \wedge \dx{#2}}}}	
\newcommand{\dV}[3]{\ensuremath{\mathrm{\dx{#1} \wedge \dx{#2} \wedge \dx{#3}}}}	
\newcommand{\J}[1]{\ensuremath{\mathrm{J_{#1}}}}
\newcommand{\vol}{\ensuremath{\Omega}}
\newcommand{\dvol}{\ensuremath{\mathrm{d}\Omega}}
\newcommand{\Map}{\ensuremath{\Phi}}

% differential operators
\newcommand{\diff}{\ensuremath{\mathrm{d}}}
\newcommand{\grad}{\ensuremath{\mathrm{grad}}}
\renewcommand{\div}{\ensuremath{\mathrm{div}}}
\newcommand{\curl}{\ensuremath{\mathrm{curl}}}

% Discrete operators
\newcommand{\Int}[1]{\ensuremath{\mathcal{I}_{#1}}}
\newcommand{\Red}[1]{\ensuremath{\mathcal{R}_{#1}}}
\newcommand{\Hodge}[1]{\ensuremath{\mathrm{H}^{(#1)}}}
\newcommand{\Laplace}[1]{\ensuremath{\Delta_d^{(#1)}}}
\newcommand{\ED}[1]{\ensuremath{\mathrm{D}^{(#1)}}}
\newcommand{\CED}[1]{\ensuremath{\mathrm{D}^{\ast(#1)}}}

% Spline definition
\newcommand{\KnotVector}[3]{\ensuremath{\vect{#1}{#2} = \left\{ \right. #3 \left.\right\}}}
\newcommand{\SplineSpace}[2]{\ensuremath{\mathcal{S}^{#1}_{\vect{#2}{}}}}

% Notation basis function
\newcommand{\basis}[2]{\ensuremath{B^{#1}_{#2} \left(\vect{x}{}\right)}}
\newcommand{\Basis}[1]{\ensuremath{\vect{B}{}^{#1}\left(\vect{x}{}\right)}}
\newcommand{\Dof}[2]{\ensuremath{\vect{\bar{#1}}{#2}}}

%%%%%%%%%
%%%%%%%%%
%%%%%%%%%

%\newcommand{\vect}[3]{\ensuremath{\boldsymbol{\mathrm{#1}}_{#2}^{#3}}}
%
%\newcommand{\curl}{\,\hspace{-1pt}{\bf curl}\hspace{-1pt}\,}
%\def\div{\,\hspace{-1pt}{\rm div}\hspace{-1pt}\,}
%\newcommand{\grad}{\,\hspace{-1pt}{\bf grad}\hspace{-1pt}\,}
%
%%%%%%%%%%
%
%\newcommand{\divhat}{\,\hspace{-1pt}{\widehat{\textup{div}}}\hspace{-1pt}\,}
%
%%%%%%%%%%
%
%\newcommand{\Lbold}{\,\hspace{-1pt}{\textup{\textbf{L}}^2}(\Omega)\hspace{-1pt}\,}
%
%\newcommand{\Hbold}{\,\hspace{-1pt}{\textup{\textbf{H}}^1}(\Omega)\hspace{-1pt}\,}
%\newcommand{\HboldKh}{\,\hspace{-1pt}{\textup{\textbf{H}}^1}(\mathcal{K}_h)\hspace{-1pt}\,}
%\newcommand{\Hboldtwo}{\,\hspace{-1pt}{\textup{\textbf{H}}^2}(\Omega)\hspace{-1pt}\,}
%\newcommand{\Hboldz}{\,\hspace{-1pt}{\textup{\textbf{H}}^1_0}(\Omega)\hspace{-1pt}\,}
%\newcommand{\Hboldn}{\,\hspace{-1pt}{\textup{\textbf{H}}^1_{\textup{\textbf{n}}}}(\Omega)\hspace{-1pt}\,}
%\newcommand{\Hboldkpone}{\,\hspace{-1pt}{\textup{\textbf{H}}^{j+1}}(\Omega)\hspace{-1pt}\,}
%\newcommand{\Hbolds}{\,\hspace{-1pt}{\textup{\textbf{H}}^s}(\Omega)\hspace{-1pt}\,}
%\newcommand{\Hboldspone}{\,\hspace{-1pt}{\textup{\textbf{H}}^{s+1}}(\Omega)\hspace{-1pt}\,}
%
%\newcommand{\Hdiv}{\,\hspace{-1pt}{\bf H}({\rm div};\Omega)\hspace{-1pt}\,}
%
%\newcommand{\Hdivz}{\,\hspace{-1pt}{\textup{\textbf{H}}_0}({\rm div};\Omega)\hspace{-1pt}\,}
%
%\newcommand{\LboldK}{\,\hspace{-1pt}{\textup{\textbf{L}}^2}(K)\hspace{-1pt}\,}
%\newcommand{\HboldK}{\,\hspace{-1pt}{\textup{\textbf{H}}^1}(K)\hspace{-1pt}\,}
%
%\newcommand{\lcurly}{\{\!\!\{}
%\newcommand{\rcurly}{\}\!\!\}}
%
%%%%%%%%%%
%
%\newcommand{\LboldD}{\,\hspace{-1pt}{\textup{\textbf{L}}^2}(D)\hspace{-1pt}\,}
%
%\newcommand{\HboldzD}{\,\hspace{-1pt}{\textup{\textbf{H}}^1_0}(D)\hspace{-1pt}\,}
%\newcommand{\HboldkD}{\,\hspace{-1pt}{\textup{\textbf{H}}^k}(D)\hspace{-1pt}\,}
%\newcommand{\HboldsD}{\,\hspace{-1pt}{\textup{\textbf{H}}^s}(D)\hspace{-1pt}\,}
%
%\newcommand{\HdivD}{\,\hspace{-1pt}{\bf H}({\rm div};D)\hspace{-1pt}\,}
%\newcommand{\HdivsD}{\,\hspace{-1pt}{\textup{\textbf{H}}^s}({\rm div};D)\hspace{-1pt}\,}
%\newcommand{\HdivzD}{\,\hspace{-1pt}{\textup{\textbf{H}}_0}({\rm div};D)\hspace{-1pt}\,}
%
%%%%%%%%%%
%
%\newcommand{\Hboldh}{\,\hspace{-1pt}{\textup{\textbf{H}}^1}(\widehat{\Omega})\hspace{-1pt}\,}
%\newcommand{\Hdivzh}{\,\hspace{-1pt}{\textup{\textbf{H}}_0}(\widehat{{\rm div}};\widehat{\Omega})\hspace{-1pt}\,}
%
%%%%%%%%%%
%
%\newcommand{\Ska}{\,\hspace{-1pt}S^k_{\boldsymbol{\alpha}}\hspace{-1pt}\,}
%\newcommand{\Skal}{\,\hspace{-1pt}S^{k-1}_{\boldsymbol{\alpha-1}}\hspace{-1pt}\,}
%
%%%%%%%%%%
%
%\newcommand{\Ciam}{\,\hspace{-1pt}C^{\infty}_{\boldsymbol{\alpha}_1,\ldots,\boldsymbol{\alpha}_d}\hspace{-1pt}\,}
%
%%%%%%%%%%
%
%\newcommand{\PiVz}{\,\hspace{-1pt}\Pi^0_{\mathcal{V}_{h}}\hspace{-1pt}\,}
%\newcommand{\PiQz}{\,\hspace{-1pt}\Pi^0_{\mathcal{Q}_{h}}\hspace{-1pt}\,}
%
%\newcommand{\PihatVz}{\,\hspace{-1pt}\widehat{\Pi}^0_{\widehat{\mathcal{V}}_{h}}\hspace{-1pt}\,}
%\newcommand{\PihatQz}{\,\hspace{-1pt}\widehat{\Pi}^0_{\widehat{\mathcal{Q}}_{h}}\hspace{-1pt}\,}
%
%%%%%%%%%%
%
%\newcommand{\boldF}{\,\hspace{-1pt}\textup{\textbf{F}}\hspace{-1pt}\,}
%
%%%%%%%%%%
%
%\newcommand{\Honebold}{\,\hspace{-1pt}\textup{\textbf{H}}^1(\Omega)\hspace{-1pt}\,}
%\newcommand{\Honeboldh}{\,\hspace{-1pt}\textup{\textbf{H}}^1(\widehat{\Omega})\hspace{-1pt}\,}
%\newcommand{\Honen}{\,\hspace{-1pt}\textup{\textbf{H}}^1_{\textup{\textbf{n}}}(\Omega)\hspace{-1pt}\,}
%\newcommand{\Honenh}{\,\hspace{-1pt}\textup{\textbf{H}}^1_{\widehat{\textup{\textbf{n}}}}(\widehat{\Omega})\hspace{-1pt}\,}
%\newcommand{\Honez}{\,\hspace{-1pt}\textup{\textbf{H}}^1_{0}(\Omega)\hspace{-1pt}\,}
%
%\newcommand{\boldPhi}{\,\hspace{-1pt}\boldsymbol{\Phi}(\Omega)\hspace{-1pt}\,}
%\newcommand{\boldPhiper}{\,\hspace{-1pt}\boldsymbol{\Phi}_{per}(\Omega)\hspace{-1pt}\,}
%\newcommand{\boldPhin}{\,\hspace{-1pt}\boldsymbol{\Phi}_{\textup{\textbf{n}}}(\Omega)\hspace{-1pt}\,}
%\newcommand{\boldPhiz}{\,\hspace{-1pt}\boldsymbol{\Phi}_0(\Omega)\hspace{-1pt}\,}